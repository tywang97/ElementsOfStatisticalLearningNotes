\chapter{Basic Expansions And Regularization}
\section{Introduction}
\noindent\textbf{Core Idea}: augment/replace the vector of inputs $X$ with additional variables. 

Let $h_m(X):\mathbb{R}^p\rightarrow\mathbb{R}$ be the $m$th transformation, 
\begin{equation*}
    f(X)=\sum_{m=1}^{M} \beta_{m} h_{m}(X)
\end{equation*}
A linear basis expansion in $X$. Once $h_m$ determined, models are linear in these new variables. 
Examples include
\begin{itemize}
\item $h_m(X)=X_m$
\item $H_m(X)=X_j^2$ or $H_m(X)=X_jX_k$. Enable us to augment inputs to achieve higher order 
Taylor expansions. 
\item $H_m(X)=\log(X_j)$, or other nonlinear transformations like $\|X\|$
\item $H_m(X)=I(L_m\le X_j\le U_m$, indicator for a region  
\end{itemize}
Let a dictionary $\mathcal{D}$ consists of all typical basis functions, we have 
\textbf{Three common approaches} to control model complexity
\begin{itemize}
\item Restriction methods, where we decide before-hand to limit the class of functions.
\item Selection methods, which adaptively scan the dictionary and include
only those basis functions $h_m$ that contribute significantly to the fit of
the model.
\item Regularization methods where we use the entire dictionary but restrict
the coefficients.
\end{itemize}

\section{Piecewise Polynomials and Splines}
Assume $X$ is one-dimensional. 
A piecewise polynomial is representing $f$ by different polynomials on different intervals. 
Let $h_m(X)=I(X\in[\xi_{m-1},\xi_{m}])$, $f(X)=\sum_m\beta_m h_m(X)$. Least square estimate
amounts to $\beta_m=\hat{Y}_m$, mean of $Y$ on the $m$th region. 

To make the function continuous, it requires $f(\xi_m^-)=f(\xi_m^+)$. A direct way is using
basis $1,X,(X-\xi_m)_+$. For smoother function, we may increase the order of local polynomials. 

\noindent\textbf{Cubic Spline}: Continuous and first/second derivatives continuous. 

As for order-$M$ spline with $K$ knots $\xi_j$, it should have continuous derivatives up to
order $M-2$. Its base is 
\begin{align*}
h_{j}(X) &=X^{j-1}, j=1, \ldots, M \\ 
h_{M+\ell}(X) &=\left(X-\xi_{\ell}\right)_{+}^{M-1}, \ell=1, \ldots, K
\end{align*}
Cubic spline: lowest-order spline without visible knot-discontinuity. \textbf{Seldom any good 
reason to go beyond that}. 

\subsection{Natural Cubic Splines}
Behavior of polynomial fit to data may be erratic near boundaries, which can be dangerous and
even more wild than global polynomials.  

\textit{Natural Cubic Splines} adds more constraints: function is linear beyond knots. A price
in bias will be paid. 

A natural cubic spline with $K$ knots can be represented by $K$ basis functions. One can start
from a basis for cubic splines and reduce it by boundary constraints. 

\section{Filtering and Feature Extraction}
Preprocess a high dimensional $x$ to some new features by $x^*=g(x)$. 

\section{Smoothing Splines}
Penalized residual sum of squares
\begin{equation}\label{PRSS}
    \operatorname{RSS}(f, \lambda)=\sum_{i=1}^{N}\left\{y_{i}-f\left(x_{i}\right)\right\}^{2}
    +\lambda \int\left\{f^{\prime \prime}(t)\right\}^{2} d t
\end{equation}
$\lambda$: fixed smoothing parameter. Here the first term measures closeness of the data
and the second penalizes curvature. 

Criterion (\ref{PRSS}) is defined on a Sobolev space, which is a space of functions for which the second
term is defined. It can be shown that it has an explicit unique finite-dimensional minimizer
with a natural spline with $N$ knots. Since the solution is a natural spline, we can write it
as
\begin{equation*}
    f(x)=\sum_{j=1}^{N} N_{j}(x) \theta_{j}
\end{equation*}
Here $N_j(x)$ is an $N$-dimensional set of basis for representing this family of natural 
splines. The criterion then reduces to
\begin{equation*}
    \operatorname{RSS}(\theta, \lambda)=(\mathbf{y}-\mathbf{N} \theta)^{T}(\mathbf{y}-\mathbf{N} 
    \theta)+\lambda \theta^{T} \mathbf{\Omega}_{N} \theta
\end{equation*}
where $\{\mathbf{N}_{ij}=N_j(x_i)\}$ and 
$\left\{\Omega_{N}\right\}_{j k}=\int N_{j}^{\prime \prime}(t) N_{k}^{\prime \prime}(t) d t$.
The solution seems to be 
\begin{equation*}
    \hat{\theta}=\left(\mathbf{N}^{T} \mathbf{N}+\lambda 
    \mathbf{\Omega}_{N}\right)^{-1} \mathbf{N}^{T} \mathbf{y}
\end{equation*}
a generalized ridge regression, and fitted spline is given by
\begin{equation*}
    \hat{f}(x)=\sum_{j=1}^{N} N_{j}(x) \hat{\theta}_{j}
\end{equation*}

\subsection{Degrees of Freedom and Smoother Matrices}
Here we discuss intuitive ways of prespecifying $\lambda$. 

A smooth spline with prechosen $\lambda$ is a linear smoother since $\hat{\theta}$ is a linear
combination of $y$. Let $\hat{\mathbf{f}}$ be the estimated values at $x_i$, 
\begin{equation*}
\hat{\mathbf{f}}=\mathbf{N}\left(\mathbf{N}^{T} \mathbf{N}
+\lambda\mathbf{\Omega}_{N}\right)^{-1}\mathbf{N}^{T}\mathbf{y}
=\mathbf{S}_{\lambda} \mathbf{y}
\end{equation*}
$\mathbf{S}_{\lambda}$ known as smoother matrix. 

Let $\mathbf{B}_\xi$ be $N\times M$ matrix of $M$ spline basis functions at $x_i$. $M \ll N$. 
Then least squares fitted spline value is defined by
\begin{align*} 
\hat{\mathbf{f}}&=\mathbf{B}_{\xi}\left(\mathbf{B}_{\xi}^{T} \mathbf{B}_{\xi}\right)^{-1}
\mathbf{B}_{\xi}^{T} \mathbf{y}=\mathbf{H}_{\xi} \mathbf{y} 
\end{align*}
$\mathbf{H}_{\xi}$ is a project operator(hat matrix). $M=tr(\mathbf{H}_\xi)$ gives the 
dimension of the projection space, which is also the number of basis functions/parameters
involved in the fit. By analogy we define \textit{effective degrees of freedom of a smoothing
spline} is
\begin{equation*}
    \mathrm{df}_{\lambda}=\operatorname{trace}\left(\mathbf{S}_{\lambda}\right)
\end{equation*}
We can then find $\lambda$ by specifying $\mathrm{df}_{\lambda}$. 

Since $\mathbf{S}_{\lambda}$ is symmetric, it has a real decomposition, and we can rewrite
it in Reinsch form
\begin{equation*}
    \mathbf{S}_{\lambda}=(\mathbf{I}+\lambda \mathbf{K})^{-1}
\end{equation*}
where $\mathbf{K}$ does not depend on $\lambda$. 
$\hat{\mathbf{f}}=\mathbf{S}_{\lambda} \mathbf{y}$ actually solves
\begin{equation*}
    \min _{\mathbf{f}}(\mathbf{y}-\mathbf{f})^{T}(\mathbf{y}-\mathbf{f})
    +\lambda \mathbf{f}^{T} \mathbf{K} \mathbf{f}
\end{equation*}
$\mathbf{K}$ known as a penalty matrix. Let $d_k$ be the $k$th eigenvalue of $\mathbf{K}$, 
the eigen-decomposition of $\mathbf{S}_{\lambda}$ is
\begin{equation*}
    \mathbf{S}_{\lambda}=\sum_{k=1}^{N} \rho_{k}(\lambda) \mathbf{u}_{k} \mathbf{u}_{k}^{T}
    =\sum_{k=1}^{N} \frac{1}{1+\lambda d_{k}} \mathbf{u}_{k} \mathbf{u}_{k}^{T}
\end{equation*}
By eigen-representation, we can also find that
\begin{itemize}
\item Eigenvectors not affected by changes in $\lambda$. 
\item $\mathbf{S}_{\lambda \mathbf{Y}}=\sum_{k=1}^{N} \mathbf{u}_{k} 
\rho_{k}(\lambda)\left\langle\mathbf{u}_{k}, \mathbf{y}\right\rangle$, hence the smoothing
spline operates by decomposing $\mathbf{y}$ w.r.t $\{\mathbf{u}_k\}$ and differentially 
shrinking the contributions using $\rho_k(\lambda)$, while $\mathbf{H}_{\xi}$ has $M$ 
eigenvalues equal to $1$ and the rest are $0$. So smoothing splines are shrinking smoothers, 
regression splines are projection smoothers. 
\item $\mathbf{u}_k$ ordered by decreasing $\rho_k(\lambda)$ increase in complexity. Higher-complexity
$\mathbf{u}_k$ are shrunk more. If domain of $X$ are periodic, $\mathbf{u}_k$ will be sin/cos. 
\item First two eigenvalues always 1(two-dimensional eigenspace of functions linear in $x$). 
\item $d_1=d_2=0$ and linear functions are not penalized. 
\item Reparametrizing using Demmler-Reinsch basis $\mathbf{u}_k$ solves
\begin{equation*}
    \min _{\boldsymbol{\theta}}\|\mathbf{y}-\mathbf{U} \boldsymbol{\theta}\|^{2}+\lambda 
    \boldsymbol{\theta}^{T} \mathbf{D} \boldsymbol{\theta},\quad \mathbf{D}=diag\{d_k\}
\end{equation*}
\end{itemize}
A smoothing spline is a local fitting method, much like the locally weighted regression. 
As $\lambda\rightarrow 0,~\mathrm{df}_{\lambda} \rightarrow N,~
\mathbf{S}_{\lambda} \rightarrow \mathbf{I}$, As 
$\lambda\rightarrow \infty,~\mathrm{df}_{\lambda} \rightarrow 2,~
\mathbf{S}_{\lambda} \rightarrow \mathbf{H}$, the hat matrix for linear regression. 

\section{Automatic Selection of the Smoothing Parameters}
\subsection{Fixing the Degree of Freedom}
$\mathrm{df}_{\lambda}=\operatorname{trace}\left(\mathbf{S}_{\lambda}\right)$ is monotone in 
$\lambda$. We can set $\mathrm{df}_{\lambda}$ to fix $\lambda$. We may try some different df 
and use some traditional criteria like F-tests, residual plots or others. 

\subsection{The Bias-Variance Tradeoff}
Since $\hat{\mathbf{f}}=\mathbf{S}_{\lambda} \mathbf{y}$, let $\operatorname{Cov}(\mathbf{y})=I$
\begin{align*}
\operatorname{Cov}(\hat{\mathbf{f}})=&\mathbf{S}_{\lambda}\operatorname{Cov}(\mathbf{y})
\mathbf{S}_{\lambda}^{T}=\mathbf{S}_{\lambda} \mathbf{S}_{\lambda}^{T}\\
\operatorname{Bias}(\hat{\mathbf{f}})=&\mathbf{f}-\mathrm{E}(\hat{\mathbf{f}})
=\mathbf{f}-\mathbf{S}_{\lambda} \mathbf{f}
\end{align*}
where $\mathbf{f}$ is the (unknown) vector of evaluations of the true $f$. 

The integrated squared prediction error (EPE) combines both bias and
variance in a single summary: 
\begin{align*} 
\operatorname{EPE}\left(\hat{f}_{\lambda}\right)&=\mathrm{E}\left(Y-\hat{f}_{\lambda}(X)\right)^{2}
=\operatorname{Var}(Y)+\mathrm{E}\left[\operatorname{Bias}^{2}\left(\hat{f}_{\lambda}(X)\right)
+\operatorname{Var}\left(\hat{f}_{\lambda}(X)\right)\right] \\ 
&=\sigma^{2}+\operatorname{MSE}\left(\hat{f}_{\lambda}\right) 
\end{align*}
EPE is a natural quantity of interest, and does create a tradeoff between bias and variance.
When we don’t know the true function, we do not have access to EPE, and need an estimate. Overall
$N-fold$ cross-validation curve could be a good estimation of EPE. 

\section{Nonparametric Logistic Regression}
Consider a general logistic model
\begin{align*}
    \log \frac{\operatorname{Pr}(Y=1 | X=x)}{\operatorname{Pr}(Y=0 | X=x)}&=f(x)\\
    \operatorname{Pr}(Y=1 | X=x)&=\frac{e^{f(x)}}{1+e^{f(x)}}
\end{align*}
To fit $f(x)$ in a smooth fashion, consider the penalized log-likelihood criterion
\begin{align*}
\ell(f ; \lambda) 
&=\sum_{i=1}^{N}\left[y_{i}\log p\left(x_{i}\right)+\left(1-y_{i}\right)\log\left(1-p\left(x_{i}\right)\right)\right]-\frac{1}{2} \lambda \int\left\{f^{\prime \prime}(t)\right\}^{2}dt\\
&=\sum_{i=1}^{N}\left[y_{i} f\left(x_{i}\right)-\log \left(1+e^{f\left(x_{i}\right)}\right)\right]-\frac{1}{2} \lambda \int\left\{f^{\prime \prime}(t)\right\}^{2} dt
\end{align*}
We can also represent $f(x)=\sum_{j=1}^{N} N_{j}(x) \theta_{j}$, and
\begin{align*} 
\frac{\partial \ell(\theta)}{\partial \theta} 
&=\mathrm{N}^{T}(\mathrm{y}-\mathrm{p})-\lambda \mathrm{\Omega} \theta \\ 
\frac{\partial^{2} \ell(\theta)}{\partial \theta \partial \theta^{T}} 
&=-\mathrm{N}^{T} \mathrm{WN}-\lambda \Omega 
\end{align*}
where $\mathrm{W}$ is a diagonal matrix of $p(x_i)(1-p(x_i))$. 
The first derivative is nonlinear in $\theta$, using Newton-Rapthson in method, we can get
update equation
\begin{align*} 
\theta^{\text {new }}
&=\left(\mathbf{N}^{T}\mathbf{W N}+\lambda\mathbf{\Omega}\right)^{-1}\mathbf{N}^{T}
\mathbf{W}\left(\mathbf{N} \theta^{\text {old }}+\mathbf{W}^{-1}(\mathbf{y}-\mathbf{p})\right)
=\left(\mathbf{N}^{T}\mathbf{W}\mathbf{N}+\lambda\mathbf{\Omega}\right)^{-1}\mathbf{N}^{T}\mathbf{W}\mathbf{z} \\
\mathbf{f}^{\text {new }}&=\mathbf{N}\left(\mathbf{N}^{T} \mathbf{W} \mathbf{N}+\lambda \mathbf{\Omega}\right)^{-1} \mathbf{N}^{T} \mathbf{W}\left(\mathbf{f}^{\text {old }}+\mathbf{W}^{-1}(\mathbf{y}-\mathbf{p})\right)
=\mathbf{S}_{\lambda, w}\mathbf{z}
\end{align*}
see that the update fits a weighted smoothing spline to the working response $\mathbf{z}$. 

\section{Multidimensional Splines}