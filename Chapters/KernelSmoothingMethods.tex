\chapter{Kernel Smoothing Methods}
Discuss regression techniques that estimates $f(X)$ over $\mathbb{R}^p$ by fitting a different
but simple model at each query point $x_0$ by only using observations close to $x_0$ such that
the resulting function $\hat{f}(X)$ is smooth. Localization is achieved by \textit{kernel} 
$K_{\lambda}(x_0,x_i)$ which assign weights by distance. $\lambda$ indicates the width of the
neighborhood. These methods require no training except for choice of $\lambda$. 

In this chapter, kernels are used for localization, but in other chapters, it may means an 
inner product in a high-dimensional feature space and is used for nonlinear modelling. 

\section{One-dimensional Kernel Smoothers}
Compared with methods like KNN, kernel methods help fitted function become smooth and continuous. 
\noindent\textbf{Nadaraya-Watson kernel weighted average}
\begin{equation*}
    \hat{f}\left(x_{0}\right)=\frac{\sum_{i=1}^{N} K_{\lambda}\left(x_{0}, x_{i}\right) 
    y_{i}}{\sum_{i=1}^{N} K_{\lambda}\left(x_{0}, x_{i}\right)}
\end{equation*}
\noindent\textbf{Epanechnikov quadratic kernel}
\begin{equation*}
    K_{\lambda}\left(x_{0}, x\right)=D\left(\frac{\left|x-x_{0}\right|}{\lambda}\right),\quad
D(t)=\left\{\begin{array}{ll}{\frac{3}{4}\left(1-t^{2}\right)} & {\text { if }|t| \leq 1} \\ 
{0} & {\text { otherwise }}\end{array}\right.
\end{equation*}
We may also replace $\lambda$ by a more general width function $h_{\lambda}(x_0)$. 

\noindent\textbf{Details to focus on}
\begin{itemize}
\item Large $\lambda$ implies lower variance but higher bias(we essentially assume
the true function is constant within the window). 
\item Metric window widths $h_\lambda(x)$
\end{itemize}

\subsection{Local Linear Regression}
Problem of smooth kernel fit: Locally-weighted averages can be badly biased on the boundaries 
because of the asymmetry of the kernel. 

Bias can be removed by fitting straight line(make first-order correction) but not constants 
locally. 

At target point $x_0$, locally weighted regression solves a separate weighted least square 
problem
\begin{equation*}
\min _{\alpha\left(x_{0}\right), \beta\left(x_{0}\right)} \sum_{i=1}^{N} K_{\lambda}
\left(x_{0}, x_{i}\right)\left[y_{i}-\alpha\left(x_{0}\right)-\beta\left(x_{0}\right) 
x_{i}\right]^{2}
\end{equation*}
which gives estimate $ \hat{f}\left(x_{0}\right)=\hat{\alpha}\left(x_{0}\right)
+\hat{\beta}\left(x_{0}\right) x_{0} $. 
\begin{itemize}
\item Fit the model in the region, but only use it to evaluate the fit at $x_0$. 
\end{itemize}

Let $b(x)^T=(1,x)$, $\mathbf{B}$ a matrix with $i$th row $b(x)^T$, 
$\mathbf{W}(x_0)=diag(K_\lambda(x_0,x_1),\cdots,K_\lambda(x_0,x_N))$, then
\begin{equation*}
\hat{f}\left(x_{0}\right)=b\left(x_{0}\right)^{T}\left(\mathbf{B}^{T} 
\mathbf{W}\left(x_{0}\right)\mathbf{B}\right)^{-1}\mathbf{B}^{T}\mathbf{W}\left(x_{0}\right) 
\mathbf{y}=\sum_{i=1}^{N} l_{i}\left(x_{0}\right) y_{i}
\end{equation*}
Local linear regression automatically modifies the kernel to correct the bias exactly to
first order, a phenomenon dubbed as automatic kernel carpentry.
\begin{align*}
\mathrm{E} \hat{f}\left(x_{0}\right)=& 
\sum_{i=1}^{N} l_{i}\left(x_{0}\right) f\left(x_{i}\right) \\
=& f\left(x_{0}\right) \sum_{i=1}^{N} l_{i}\left(x_{0}\right)+f^{\prime}\left(x_{0}\right)
\sum_{i=1}^{N}\left(x_{i}-x_{0}\right) l_{i}\left(x_{0}\right)+\frac{f^{\prime \prime}
\left(x_{0}\right)}{2} \sum_{i=1}^{N}\left(x_{i}-x_{0}\right)^{2} l_{i}\left(x_{0}\right)
+R
\end{align*}
where $R$ involves third and higher order derivatives of $f$. 
It can be shown that in local linear regression $\sum_{i=1}^{N} l_{i}\left(x_{0}\right)=1$
and $\sum_{i=1}^{N}\left(x_{i}-x_{0}\right) l_{i}\left(x_{0}\right)=0$. Therefore bias
$E\hat{f}(x_0)-f(x_0)$ only depends on quadratic and higher order terms in $f$ expansion. 

\subsection{Local Polynomial Regression}
For any degree $d$, we can also fit local polynomial fits
\begin{equation*}
    \min _{\alpha\left(x_{0}\right), \beta_{j}\left(x_{0}\right), j=1, \ldots, d} 
    \sum_{i=1}^{N} K_{\lambda}\left(x_{0}, 
    x_{i}\right)\left[y_{i}-\alpha\left(x_{0}\right)-\sum_{j=1}^{d} 
    \beta_{j}\left(x_{0}\right) x_{i}^{j}\right]^{2}
\end{equation*}
Such expansion will tell us the bias only have components of degree $d+1$ and higher. 
Increased variance will be paid for this bias reduction. 
\begin{itemize}
\item Local linear fits help bias at boundaries at a modest cost in variance. 
Local quadratic fits do little at the boundaries for bias, but increase the variance a lot.
\item Local quadratic fits tend to be most helpful in reducing bias due to
curvature in the interior of the domain.
\item Asymptotic analysis suggest that local polynomials of odd degree
dominate those of even degree. This is largely due to the fact that
asymptotically the MSE is dominated by boundary effects.
\end{itemize}

\section{Selecting the Width of the Kernel}
\noindent\textbf{A natural bias-variance tradeoff}
\begin{itemize}
\item If the window is too narrow, variance will be very large, bias tend to be small. 
\item If the window is too narrow, variance will be small because of the effects of averaging, 
but bias will be higher. 
\end{itemize}
Similar arguments for local linear: when $\lambda$ goes from too small to too large, estimates
goes from piecewise-linear interpolation to global OLS. 

\section{Local Regression in $\mathbb{R}^p$}
Let $b(X)$ be a vector of polynomials terms in $X$ of maximum degree $d$, at each $x_0\in
\mathbb{R}^p$ we solves
\begin{equation*}
    \min _{\beta\left(x_{0}\right)} \sum_{i=1}^{N} K_{\lambda}\left(x_{0}, x_{i}\right)
    \left(y_{i}-b\left(x_{i}\right)^{T} \beta\left(x_{0}\right)\right)^{2}, \quad
    K_{\lambda}\left(x_{0}, x\right)=D\left(\frac{\left\|x-x_{0}\right\|}{\lambda}\right)
\end{equation*}
It makes more sense to \textbf{standardize each predictor} since norm depends on unit. 

When dimension becomes higher, curse of dimensionality makes fraction of points close to $1$
increase to one, which makes boundary effects a larger problem. Local regreesion will be much 
less useful when dimensions much higher than two or three. 

\section{Structured Local Regression Models in $\mathbb{R}^p$}
When dimension to sample-size ratio is unfavorable, local regression would be unhelpful without
some structural assumptions. 

\subsection{Structured Kernels}
A general approach is to use a positive semidefinite matrix $\mathbf{A}$ to weigh the different 
coordinates. 
\begin{equation*}
K_{\lambda,A}(x_0,x)=D\left(\frac{(x-x_0)^T\mathbf{A}(x-x_0)}{\lambda}\right)
\end{equation*}
Entire directions can be controlled by imposing restrictions on $\mathbf{A}$. For example
changing $A_{jj}$ when $\mathbf{A}$ is diagonal. 
For data like digital signal and images, predictors might be highly correlated, and we may
use covariance to tailor $\mathbf{A}$ to make it focuses less on  high-frequency contrasts. 

\subsection{Structured Regreesion Functions}
Trying to fit $E(Y|X)=f(X_1,X_2,\ldots,X_p)$ in $\mathbb{R}^p$, it is natural to use 
analysis-of-variance(ANOVA) decompositions
\begin{equation*}
f(X_1,X_2,\ldots,X_p)=\alpha+\sum_jg_j(X_j)+\sum_{k<l}g_{kl}(X_k,X_l)+\cdots
\end{equation*}
and then introduce structures to eliminate high order terms e.g. additive model 
$f(X)=\alpha+\sum_{j=1}^pg_j(X_j)$(we can estimate $g_k$ by local regression of 
$Y-\sum_{j\neq k}g_j(X_j)$ on $X_k$. This is done for each function in term repeatedly until
convergence). 

A special case is \textit{varying coefficient models}. For example, we divide the $p$ predictors
in $X$ into $(X_1,X_2,\ldots,X_q)$ with $q<p$, and the remainder in vector $Z$, then
\begin{equation*}
f(X)=\alpha(Z)+\beta_1(Z)X_1+\cdots+\beta_q(Z)X_q
\end{equation*}
It is natural to fit such a model by locally weighted least squares: 
\begin{equation*}
\min_{\alpha(z_0),\beta(z_0)}\sum_{i=1}^NK_{\lambda}(z_0,z_i)(y_i-\alpha(z_0)-x_{1i}\beta_1(z_0)
-\cdots-x_{qi}\beta_q(z_0))^2
\end{equation*}

\section{Local Likelihood and Other Methods}
Any parametric model can be made local if the fitting method accommodates observation weights. 
\begin{itemize}
\item Generalized linear models including logistic and log-linear models. Let $\theta_i=
\theta(x_i)=x_i^T\beta$, inference for $\beta$ based on the log-likelihood 
$l(\beta)=\sum_{i=1}^Nl(y_i,x_i^T\beta)$, or model it more flexibly by making likelihood local
\begin{equation*}
l(\beta(x_0))=\sum_{i=1}^NK_{\lambda}(x_0,x_i)l(y_i,x_i^T\beta(x_0))
\end{equation*}
Many likelihood models involve the covariates in a linear fashion. Local likelihood allows a 
relaxation from a globally linear model to one that is local linear. 
\item Autoregressive time seriesmodels fitted with local least squares allows the model to 
vary according to the short-term history of the series. 
\end{itemize} 

\section{Kernel Density Estimation and Classification}
Kernel density estimation: an unsupervised learning procedure, leading naturallbecausey to a simple 
family of procedures for nonparametric classification. 

\subsection{Kernel Density Estimation}
A natural estimate for pdf $f_X(x)$
\begin{equation*}
\hat{f}_X(x_0)=\frac{\#x_i\in\mathcal{N}(x_0)}{N\lambda}
\end{equation*}
where $\mathcal{N}(x_0)$ is a neighborhood with width $\lambda$. This estimate is bumpy, and 
a smooth \textit{Parzen} estimate is preferred
\begin{equation*}
\hat{f}_X(x_0)=\frac{1}{N\lambda}\sum_{i=1}^NK_{\lambda}(x_0,x_i)
\end{equation*}
because it counts observations close to $x_0$ with weights that decrease with distance. 
A popular choice is Gaussian Kernel $K_{\lambda}(x_0,x)=\phi(\abs{x-x_0}/\lambda)$. 
Let $\phi_{\lambda}$ be pdf of $N(0,\lambda^2)$, then 
\begin{equation*}
\hat{f}_X(x)=\frac{1}{N}\sum_{i=1}^N\phi_{\lambda}(x-x_i)=(\hat{F}*\lambda)(x)
\end{equation*}
where $\hat{F}$ is the sample empirical distribution, which means put mass $1/N$ at each $x_i$.
In $\hat{f}$ we smoothed $\hat{F}$ by adding independent Gaussian noise.   

\subsection{Kernel Density Classification}
Using nonparametric density estimates for classification using Bayes. 
Suppose for a $J$ class problem we fit density estimates $\hat{f}_j(X)$ in each of the class
and we have class priors $\hat{\pi}_j$(sample proportion). Then
\begin{equation*}
\hat{Pr}(G=j|X=x_0)=\frac{\hat{\pi}_j\hat{f}_j(x_0)}{\sum_{k=1}^J\hat{\pi}_k\hat{f}_k(x_0)}
\end{equation*}
If classification is the ultimate goal, then learning the separate class densities
well may be unnecessary, and can in fact be misleading. In learning the separate densities from data,
one might decide to settle for a rougher, high-variance fit to capture these
features, which are irrelevant for the purposes of estimating the posterior
probabilities. In fact, if classification is the ultimate goal, then we need only
to estimate the posterior well near the decision boundary. 

\subsection{The Naive Bayes Classifier}
High $p$ makes density estimation unattractive. Naive Bayes assumes that given $G=j$, features
$X_k$ independent, 
\begin{equation*}
f_j(X)=\prod_{k=1}^pf_{jk}(X_k)
\end{equation*}
While this assumption is generally not true, it does simplify the estimation: 
\begin{itemize}
\item Individual class-conditional marginal density $f_{jk}$ can each be estimated separately
using one-dim kernel density estimates. 
\item If $X_j$ is discrete, an appropriate histogram estimate can then be used. 
\end{itemize}
Naive Bayes often outperform more sophisticated approaches. 

We can get a generalized additive model by deriving logit-transform. 

\section{Radial Basis Functions and Kernels}
Radial Basis combine ideas of basis functions and kernel methods by treating 
$K_{\lambda}(\xi,x)$ as basis functions: 
\begin{equation*}
f(x)=\sum_{j=1}^MK_{\lambda_j}(\xi_j,x)\beta_j=\sum_{j=1}^MD\left(\frac{\|x-\xi)j\|}{\lambda}
\right)\beta_j
\end{equation*}
where each basis element is indexed by a location or prototype parameter $\xi_j$ and a scale
parameter $\lambda_j$. A popular choice is the standard Gaussian. For simplicity we use least
squares to fit $\{\lambda_j,\xi_j,\beta_j\}$. 
\begin{itemize}
\item Optimize sum of squares with respect to all parameters is referred to as an RBF network, 
a nonconvex problem with multiple local minima. 
\item Given $\{\lambda_j,\xi_j\}$, estimation of $\beta_j$ is an OLS problem. Often
$\lambda_j,\xi_j$ are chosen in an unsupervised way using $X$ distribution alone. One example
is fitting a Gaussian mixture density model to train $x_i$ and provides both centers $\xi_j$ 
and scales $\lambda_j$. We can also use clustering methods to locate $\xi_j$ and let $\lambda$
be a hyper-parameter.  
\end{itemize}

Reducing parameters by $\lambda_j=\lambda$ may create holes: some regions no kernel have support. 
Renormalized radial basis functions will solve the problem
\begin{equation*}
h_j(x)=\frac{D(\|x-\xi_j\|)/\lambda}{\sum D(\|x-\xi_j\|)/\lambda}
\end{equation*}

\section{Mixture Models for Density Estimation and Classification}
\begin{equation*}
f(x)=\sum_{m=1}^M\alpha_m\phi(x;\mu_m,\mathbf{\Sigma}_m),~\sum_m\alpha_m=1
\end{equation*}
Usually fit by maximum likelihood like EM. 
\begin{itemize}
\item When $\mathbf{\Sigma}_m$ is diagonal, it has the form of radial expansion.
\item If in addition $\sigma_m=\sigma$ and $M\rightarrow N$, then $\hat{\alpha}_m=1/N,~
\hat{\mu}_m=x_m$. 
\end{itemize}

\section{Computational Considerations}
Kernel and local regression and density estimation are memory-based methods:
the model is the entire training data set, and the fitting is done at
evaluation or prediction time. For many real-time applications, this can
make this class of methods infeasible. 

Basis function methods: at least $O(NM^2+M^3)$. 

Smoothing parameter $\lambda$: determined offline at $O(N)$. 