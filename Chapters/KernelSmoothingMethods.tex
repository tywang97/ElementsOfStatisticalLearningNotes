\chapter{Kernel Smoothing Methods}
Discuss regression techniques that estimates $f(X)$ over $\mathbb{R}^p$ by fitting a different
but simple model at each query point $x_0$ by only using observations close to $x_0$ such that
the resulting function $\hat{f}(X)$ is smooth. Localization is achieved by \textit{kernel} 
$K_{\lambda}(x_0,x_i)$ which assign weights by distance. $\lambda$ indicates the width of the
neighborhood. These methods require no training except for choice of $\lambda$. 

In this chapter, kernels are used for localization, but in other chapters, it may means an 
inner product in a high-dimensional feature space and is used for nonlinear modelling. 

\section{One-dimensional Kernel Smoothers}
Compared with methods like KNN, kernel methods help fitted function become smooth and continuous. 
\noindent\textbf{Nadaraya-Watson kernel weighted average}
\begin{equation*}
    \hat{f}\left(x_{0}\right)=\frac{\sum_{i=1}^{N} K_{\lambda}\left(x_{0}, x_{i}\right) 
    y_{i}}{\sum_{i=1}^{N} K_{\lambda}\left(x_{0}, x_{i}\right)}
\end{equation*}
\noindent\textbf{Epanechnikov quadratic kernel}
\begin{equation*}
    K_{\lambda}\left(x_{0}, x\right)=D\left(\frac{\left|x-x_{0}\right|}{\lambda}\right),\quad
D(t)=\left\{\begin{array}{ll}{\frac{3}{4}\left(1-t^{2}\right)} & {\text { if }|t| \leq 1} \\ 
{0} & {\text { otherwise }}\end{array}\right.
\end{equation*}
We may also replace $\lambda$ by a more general width function $h_{\lambda}(x_0)$. 

\noindent\textbf{Details to focus on}
\begin{itemize}
\item Large $\lambda$ implies lower variance but higher bias(we essentially assume
the true function is constant within the window). 
\item Metric window widths $h_\lambda(x)$
\end{itemize}

\subsection{Local Linear Regression}