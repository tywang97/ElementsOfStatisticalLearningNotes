\chapter{Kernel Smoothing Methods}
Discuss regression techniques that estimates $f(X)$ over $\mathbb{R}^p$ by fitting a different
but simple model at each query point $x_0$ by only using observations close to $x_0$ such that
the resulting function $\hat{f}(X)$ is smooth. Localization is achieved by \textit{kernel} 
$K_{\lambda}(x_0,x_i)$ which assign weights by distance. $\lambda$ indicates the width of the
neighborhood. These methods require no training except for choice of $\lambda$. 

In this chapter, kernels are used for localization, but in other chapters, it may means an 
inner product in a high-dimensional feature space and is used for nonlinear modelling. 

\section{One-dimensional Kernel Smoothers}
Compared with methods like KNN, kernel methods help fitted function become smooth and continuous. 
\noindent\textbf{Nadaraya-Watson kernel weighted average}
\begin{equation*}
    \hat{f}\left(x_{0}\right)=\frac{\sum_{i=1}^{N} K_{\lambda}\left(x_{0}, x_{i}\right) 
    y_{i}}{\sum_{i=1}^{N} K_{\lambda}\left(x_{0}, x_{i}\right)}
\end{equation*}
\noindent\textbf{Epanechnikov quadratic kernel}
\begin{equation*}
    K_{\lambda}\left(x_{0}, x\right)=D\left(\frac{\left|x-x_{0}\right|}{\lambda}\right),\quad
D(t)=\left\{\begin{array}{ll}{\frac{3}{4}\left(1-t^{2}\right)} & {\text { if }|t| \leq 1} \\ 
{0} & {\text { otherwise }}\end{array}\right.
\end{equation*}
We may also replace $\lambda$ by a more general width function $h_{\lambda}(x_0)$. 

\noindent\textbf{Details to focus on}
\begin{itemize}
\item Large $\lambda$ implies lower variance but higher bias(we essentially assume
the true function is constant within the window). 
\item Metric window widths $h_\lambda(x)$
\end{itemize}

\subsection{Local Linear Regression}
Problem of smooth kernel fit: Locally-weighted averages can be badly biased on the boundaries 
because of the asymmetry of the kernel. 

Bias can be removed by fitting straight line(make first-order correction) but not constants 
locally. 

At target point $x_0$, locally weighted regression solves a separate weighted least square 
problem
\begin{equation*}
\min _{\alpha\left(x_{0}\right), \beta\left(x_{0}\right)} \sum_{i=1}^{N} K_{\lambda}
\left(x_{0}, x_{i}\right)\left[y_{i}-\alpha\left(x_{0}\right)-\beta\left(x_{0}\right) 
x_{i}\right]^{2}
\end{equation*}
which gives estimate $ \hat{f}\left(x_{0}\right)=\hat{\alpha}\left(x_{0}\right)
+\hat{\beta}\left(x_{0}\right) x_{0} $. 
\begin{itemize}
\item Fit the model in the region, but only use it to evaluate the fit at $x_0$. 
\end{itemize}

Let $b(x)^T=(1,x)$, $\mathbf{B}$ a matrix with $i$th row $b(x)^T$, 
$\mathbf{W}(x_0)=diag(K_\lambda(x_0,x_1),\cdots,K_\lambda(x_0,x_N))$, then
\begin{equation*}
\hat{f}\left(x_{0}\right)=b\left(x_{0}\right)^{T}\left(\mathbf{B}^{T} 
\mathbf{W}\left(x_{0}\right)\mathbf{B}\right)^{-1}\mathbf{B}^{T}\mathbf{W}\left(x_{0}\right) 
\mathbf{y}=\sum_{i=1}^{N} l_{i}\left(x_{0}\right) y_{i}
\end{equation*}
Local linear regression automatically modifies the kernel to correct the bias exactly to
first order, a phenomenon dubbed as automatic kernel carpentry.
\begin{align*}
\mathrm{E} \hat{f}\left(x_{0}\right)=& 
\sum_{i=1}^{N} l_{i}\left(x_{0}\right) f\left(x_{i}\right) \\
=& f\left(x_{0}\right) \sum_{i=1}^{N} l_{i}\left(x_{0}\right)+f^{\prime}\left(x_{0}\right)
\sum_{i=1}^{N}\left(x_{i}-x_{0}\right) l_{i}\left(x_{0}\right)+\frac{f^{\prime \prime}
\left(x_{0}\right)}{2} \sum_{i=1}^{N}\left(x_{i}-x_{0}\right)^{2} l_{i}\left(x_{0}\right)
+R
\end{align*}
where $R$ involves third and higher order derivatives of $f$. 
It can be shown that in local linear regression $\sum_{i=1}^{N} l_{i}\left(x_{0}\right)=1$
and $\sum_{i=1}^{N}\left(x_{i}-x_{0}\right) l_{i}\left(x_{0}\right)=0$. Therefore bias
$E\hat{f}(x_0)-f(x_0)$ only depends on quadratic and higher order terms in $f$ expansion. 

\subsection{Local Polynomial Regression}
For any degree $d$, we can also fit local polynomial fits
\begin{equation*}
    \min _{\alpha\left(x_{0}\right), \beta_{j}\left(x_{0}\right), j=1, \ldots, d} 
    \sum_{i=1}^{N} K_{\lambda}\left(x_{0}, 
    x_{i}\right)\left[y_{i}-\alpha\left(x_{0}\right)-\sum_{j=1}^{d} 
    \beta_{j}\left(x_{0}\right) x_{i}^{j}\right]^{2}
\end{equation*}
Such expansion will tell us the bias only have components of degree $d+1$ and higher. 
Increased variance will be paid for this bias reduction. 
\begin{itemize}
\item Local linear fits help bias at boundaries at a modest cost in variance. 
Local quadratic fits do little at the boundaries for bias, but increase the variance a lot.
\item Local quadratic fits tend to be most helpful in reducing bias due to
curvature in the interior of the domain.
\item Asymptotic analysis suggest that local polynomials of odd degree
dominate those of even degree. This is largely due to the fact that
asymptotically the MSE is dominated by boundary effects.
\end{itemize}

\section{Selecting the Width of the Kernel}
\noindent\textbf{A natural bias-variance tradeoff}
\begin{itemize}
\item If the window is too narrow, variance will be very large, bias tend to be small. 
\item If the window is too narrow, variance will be small because of the effects of averaging, 
but bias will be higher. 
\end{itemize}
Similar arguments for local linear: when $\lambda$ goes from too small to too large, estimates
goes from piecewise-linear interpolation to global OLS. 

\section{Local Regression in $\mathbb{R}^p$}
Let $b(X)$ be a vector of polynomials terms in $X$ of maximum degree $d$, at each $x_0\in
\mathbb{R}^p$ we solves
\begin{equation*}
    \min _{\beta\left(x_{0}\right)} \sum_{i=1}^{N} K_{\lambda}\left(x_{0}, x_{i}\right)
    \left(y_{i}-b\left(x_{i}\right)^{T} \beta\left(x_{0}\right)\right)^{2}, \quad
    K_{\lambda}\left(x_{0}, x\right)=D\left(\frac{\left\|x-x_{0}\right\|}{\lambda}\right)
\end{equation*}
It makes more sense to \textbf{standardize each predictor} since norm depends on unit. 

When dimension becomes higher, curse of dimensionality makes fraction of points close to $1$
increase to one, which makes boundary effects a larger problem. Local regreesion will be much 
less useful when dimensions much higher than two or three. 

\section{Structured Local Regression Models in $\mathbb{R}^p$}
