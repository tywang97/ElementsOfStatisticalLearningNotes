\chapter{Overview of Supervised Learning}

\section{Least Squares and Nearest Neighbors}
Consider two scenario: 
\begin{enumerate}
	\item \textbf{(Better on linear regression)} The training data was generated from bivariate Gaussian distributions with uncorrelated components and different means. 
	\item \textbf{(Better on nearest-neighbor)} The training data came from a mixture of multiple low-variance Gaussian distributions, with individual means themselves distributed as a Gaussian. 
\end{enumerate}
\subsection{Linear Regression Models and Least Squares}
Given $X=(X_1,X_2,...,X_P)$, predict the output $Y$ via model
\begin{equation*}
\hat{Y}=\hat{\beta_0}+\sum_{j=1}^{p}X_j\hat{\beta_j}=X^T\hat{\beta}
\end{equation*}
\begin{itemize}
	\item When $\beta_0=0$, hyperplane $(X,\hat{Y})$ forms a subspace. Otherwise, it is an affine set. 
\end{itemize}

\textbf{Least squares: }$RSS(\beta)=(Y-X\beta)^T(Y-X\beta)=||Y-X\beta||_2^2$
\begin{itemize}
	\item Quatratic$\Rightarrow$Minimum always exists. 
\end{itemize}

\textbf{Unique solution: }$\hat{\beta}=(X^TX)^{-1}X^TY$
\begin{itemize}
	\item Calculated by differentiating $RSS(\beta)$ with respect to $\beta$. 
	\item Minimizing $RSS(\beta)$ $\Leftrightarrow$ choosing $\hat{\beta}$ so that $Y-\hat{Y}$ is orthogonal to the subspace spanned by column vectors of $X$ $\Leftrightarrow$ $\displaystyle\frac{\partial RSS}{\partial\beta}=-2X^T(Y-X\beta)=0$.
	\item \textbf{Do not need a large data set to fit}.  
\end{itemize}

\subsection{Nearest-Neighbor Methods}
Use observations in the training set $\mathcal{T}$ closest to $x$ to form $\hat{Y}$. 
\begin{equation*}
	\hat{Y}(x)=\frac{1}{k}\sum_{x_i\in N_k(x)}y_i
\end{equation*}
\begin{itemize}
	\item Effective number of parameters of kNN is $N/k$ (\href{https://stats.stackexchange.com/questions/357524/why-is-n-k-the-effective-number-of-parameters-in-k-nn}{explanation}). 
	\item Error on train data should be an increasing function of $k$, and equals $0$ for $k=1$. 
	\item Unnecessarily noisy for Gaussian data. 
\end{itemize}
\subsection{From Least Squares to Nearest Neighbors}
\textbf{Linear boundary from least squares}: smooth, relies heavily on the linear assumption. Low variance and high bias. 

\textbf{kNN}: Does not rely on assumptions about underlying data. Any subregion depends on input points and their position. High variance and low bias. 
\begin{itemize}
	\item 1-nearest-neighbor captures a large percentage of the market for low-dimensional problems. 
\end{itemize}

Ways in which these procedures are enhanced
\begin{itemize}
	\item Kernel methods use weights that decrease smoothly to zero with distance from the target method instead of 0/1 in kNN. 
	\item In high dimensional spaces the distance kernels are modified to emphasize some variables more than others. 
	\item Local regression fits linear models by locally weighted least squares rather than fitting constants locally. 
	\item Basis expansion of the inputs to improve complexity of linear model. 
	\item Projection pursuit/NN models: nonlinear transform of linear models
\end{itemize}

\section{Statistical Decision Theory}
\subsection{Quantitative output}
Let $X\in\mathbb{R}^p, Y\in\mathbb{R}$ with joint distribution $Pr(X,Y)$. 

\textbf{Goal}: Find a function $f(X)$ for predicting $Y$, which requires a \textit{loss function} $L(Y, f(X))$. 

\textbf{Squared error loss}: 
\begin{align*}
	L(Y,f(X))=&(Y-f(X))^2\\
	EPE(f)=&E(Y-f(X))^2=\int[y-f(x)]^2Pr(dx,dy)\\
	=&\int[y-f(x)]^2Pr(dy|dx)Pr(dx)=E_XE_{Y|X}([Y-f(X)]^2|X)
\end{align*}
where EPE stands for Expected Prediction Error. I$u(x,t)=0$ when $x<0$ t suffices to minimize EPE pointwise
\begin{equation*}
\end{equation*}
Thus when the measure is squared error, $E(Y|X=x)$ is the best solution. 

~

\textbf{kNN}: 
\begin{equation*}
	\hat{f}(x)=\text{Avg}(y_i|x_i\in N_k(x))
\end{equation*}
\begin{itemize}
	\item Expectation is approximated by averaging
	\item Conditioning at a point is relaxed to a region
	\item As $N,k\rightarrow\infty$ such that $k/N\rightarrow0$, $\hat{f}(x)\rightarrow E(Y|X=x)$
	\begin{itemize}
		\item Since we do not have many samples in most cases, we can usually get a more stable model if some more structured model is appropriate. 
	\end{itemize}
	\item Assumes $f(x)$ is well approximated by a locally constant function. 
\end{itemize}

~

\textbf{Linear Regression}:Let $f(x)=x^T\beta$ and differentiating $E(Y-f(X))^2$, we have optimal $\beta$
\begin{equation*}
	\beta=[E(XX^T)]^{-1}E(XY)
\end{equation*}
The least squares solution just replace the expectation by averages on the training data. 
\begin{itemize}
	\item Assumes $f(x)$ is well approximated by a globally linear function
\end{itemize}

~

If we use $L_1$ loss instead of $L_2$, the solution will becomes the conditional median
\begin{equation*}
	\hat{f}(x)=\text{median}(Y|X=x)
\end{equation*}
\begin{itemize}
	\item A different measure of location
	\item More robust than conditional mean
\end{itemize}

\subsection{Categorical output}
Let the categorical variable be $G$, $K=card(\mathcal{G})$, $K\times K$ matrix $L$ be its loss function, $L_{ii}=0, L_{ij}\ge0$, $L(k,l)$ be the cost of wrong prediction(0/1 in most cases). 
\begin{align*}
	EPE=&E[L(G, \hat{G}(X))]\\
	=&E_X\sum_{k=1}^{K}L[\mathcal{G}_k, \hat{G}(X)]Pr(\mathcal{G}_k|X)\\
	\hat{G}(x)=&argmin_{g\in\mathcal{G}}\sum_{k=1}^{K}L[\mathcal{G}_k, g]Pr(\mathcal{G}_k|X=x)\\
\end{align*}
With 0-1 loss, it can by simplified as
\begin{align*}
	\hat{G}(x)=&argmin_{g\in\mathcal{G}}[1-Pr(g|X=x)]\\
	=&\mathcal{G}_k \text{ if } Pr(\mathcal{G}_k|X=x)=\max_{g\in\mathcal{G}}Pr(g|X=x)	
\end{align*}
It is known as \textit{Bayes classifier}, says we classify the most probable class with conditional distribution $Pr(G|X)$. The error rate here is called \textit{Bayes rate}. 

\section{Local Methods in High Dimensions}
In \textbf{low dimension space}, kNN can always find the theoretically optimal conditional expectaion with a large training set. 
\begin{itemize}
	\item Not work in high dimentional data: \textit{Curse of dimensionality}
\end{itemize}
\begin{exmp}
	Consider $N$ data points uniformly distributed on a unit ball, the median distance from the origin to the closet data point is
	\begin{equation*}
	d(p,N)=(1-\frac{1}{2}^{1/N})^{1/p}
	\end{equation*}
	\begin{itemize}
		\item Let $N=500,p=10\Rightarrow d(p,N)=0.52\Rightarrow$Most points are closer to boundary but not other points$\Rightarrow$Must extrapolate from samples but not interpolate between them. 
		\item Sampling density is proportional to $N^{1/p}$
	\end{itemize}
\end{exmp}
\begin{exmp}
	Consider 1000 $x_i$ from $U([-1,1]^p)$, $Y=f(X)=exp(-8||X||^2)$, then use 1-NN to predict $y_0$ at $x_0=0$, we have \textit{bias-variance decomposition}
	\begin{align*}
		MSE(x_0)=&E[f(x_0)-\hat{y}_0]^2=E[\hat{y}_0-E[\hat{y}_0]]^2+[E[\hat{y}_0]-f(x_0)]^2\\
		=&Var(\hat{y}_0)+Bias^2(\hat{y}_0)
	\end{align*}
	When dimension increase, we will find that distance of the closet point to $0$ increase. The bias will also significantly increase. 
	\begin{itemize}
		\item Complexity of functions of many variables grow exponentially with the dimension--Need exponential growth of the size of the training set to maintain accuracy. 
	\end{itemize}
	If we use function which only cares about a few dimensions like $f(X)=\frac{1}{2}(X_1+1)^3$, the bias will be stable, but the variance will increase a lot. 
\end{exmp}

~

\textbf{Linear model}

Let $Y=X^T\beta+\epsilon,\epsilon\sim N(0,\sigma^2)$, than for test point $x_0$,  $\hat{y}_0=x_0^T\hat{\beta}_0=x_0^T\beta+\sum_{i=1}^Nl_i(x_i)\epsilon_i$, $l_i$ is the $i$th element of $X(X^TX)^{-1}x_0$. Since the model under the least squares is unbiased, we have
\begin{align*}
	EPE(x_0)=&E_{y_0|x_0}E_{\mathcal{T}}(y_0-\hat{y}_0)^2\\
	=&Var(y_0|x_0)+E_{\mathcal{T}}[\hat{y}_0-E_{\mathcal{T}}\hat{y}_0]^2+[E_{\mathcal{T}}\hat{y}_0-x_0^T\beta]^2\\
	=&Var(y_0|x_0)+Var_{\mathcal{T}}(\hat{y}_0)+Bias^2(\hat{y}_0)\\
	=&\sigma^2+E_{\mathcal{T}}x_0^T(X^TX)^{-1}x_0\sigma^2+0^2
\end{align*}
When $N$ is large and $\mathcal{T}$ is selected at random, assume $E[X]=0$, then $X^TX\rightarrow NCov(X)$
\begin{align*}
	E_{x_0}EPE(x_0)\sim&E_{x_0}x_0^TCov(X)^{-1}x_0\sigma^2/N+\sigma^2\\
	=&trace[Cov(X)^{-1}Cov(x_0)]\sigma^2/N+\sigma^2\\
	=&\sigma^2(p/N)
\end{align*}
Thus expected EPE increases as a linear function of $p$ with slope $\sigma^2/N$. 

\section{Statistical Models, Supervised Learning and Function Approximation} 
Goal: find a $\hat{f}(x)$ of $f(x)$. Squared error lead to $f(x)=E(Y|X=x)$, but they may fail
\begin{itemize}
	\item If the dimension is too high, nearest neighbors can not be close to the target and cause large error. 
	\item If the structure is known, this can be used to reduce bias and variance. 
\end{itemize}
\subsection{A Statistical Model for the Joint Distribution $Pr(X,Y)$}
Suppose our data comes from $Y=f(X)+\epsilon$







