\chapter{Linear Methods for Classification}
\section{Introduction}
Decision boundaries are linear in this chapter. 

Suppose there are $K$ classes in discrete set $\mathcal{G}$, and the fitted linear model is 
$\hat{f}_k(x)=\hat{\beta}_{k0}+\hat{\beta}_k^Tx$. The decision boundary for class $k$ 
and $l$ should then be the set where $\hat{f}_k(x)=\hat{f}_l(x)$, which is an affine
set or hyperplane. This approach is a member of a class of methods that model 
\textit{discriminant functions} $\delta_k(x)$ for each class. Methods that model
$Pr(G=k|X=x)$ are also in this clear. All we require is some monotone transformation
of $\delta_k$ or $Pr(G=k|X=x)$ be linear for the boundaries to be linear. For example, 
we consider \textit{logit transformation} $log[p/(1-p)]$, we can see that
\begin{align*}
Pr(G=1|X=x)=&\frac{exp(\beta_0+\beta^Tx)}{1+exp(\beta_0+\beta^Tx)}\\
Pr(G=2|X=x)=&\frac{1}{1+exp(\beta_0+\beta^Tx)}\\
log\frac{Pr(G=1|X=x)}{Pr(G=2|X=x)}=&\beta_0+\beta^Tx
\end{align*}

\section{Linear Regression of an Indication Matrix}
$Y=(Y_1,...,Y_k)$, $Y_k=1(G=k)$. Then 
\begin{equation*}
\hat{Y}=X(X^TX)^{-1}X^TY
\end{equation*}
Then for a new observation $x$, 
\begin{equation*}
\hat{f}(x)^T=(1,x^T)\hat{B},\quad \hat{G}(x)=argmax_{k\in\mathcal{G}}\hat{f}_k(x)
\end{equation*}
We can verify that $\sum_{k\in\mathcal{G}}\hat{f}_k(x)=1$, however, in real world, 
$\hat{f}_k(x)=1$ can be larger than 1 or negative. We may solve the problem by basis 
expansions of the inputs. 

A more simple way is to construct targets $t_k$ for each class($t_k$ the $k$th row of 
$I_k$). Then we try to fit
\begin{equation*}
\min_B\sum_{i=1}^N||y_i-[(1,x_i^T)B]^T||^2,\quad \hat{G}(x)=argmin_k||\hat{f}(x)-t_k||^2
\end{equation*}

When $K$ becomes large, classes can be masked by others because of the regression model. 
Polynomials up to degree $K-1$ might be needed tro solve them. The worst complexity 
could be $O(p^{K-1})$. 

\section{Linear Discriminant Analysis}
We need to know $Pr(G|X)$ for optimal classification. Suppose $f_k(x)$ is the density
of $X$ in class $G=k$, $\pi_k$ be the prior probability of class $k$, $\sum \pi_k=1$. 
Then by Bayes
\begin{equation*}
Pr(G=k|X=x)=\frac{f_k(x)\pi_k}{\sum_{l=1}^Kf_l(x)\pi_l}
\end{equation*}
Many techniques are based on models for the class densities. 
\begin{itemize}
\item Linear/Quadratic discriminant analysis: Gaussian
\item Nonlinear decision boundaries: Mixture of Gaussians
\item General nonparametric density estimates for each class allow the most flexibility. 
\item Naive Bayes assumes inputs are conditionally independent in each class. 
\end{itemize}
Suppose each class density as $N(\mu_j,\Sigma_k)$. LDA arises when $\Sigma_k=\Sigma$. 
Then we can see that
\begin{align*}
log\frac{Pr(G=k|X=x)}{Pr(G=l|X=x)}=&\log\frac{f_k(x)}{f_l(x)}+\log(\pi_k){\pi_l}\\
=&\log\frac{\pi_k}{\pi_l}-\frac{1}{2}(\mu_k+\mu_l)^T\Sigma^{-1}(\mu_k-\mu_l)+
x^T\Sigma^{-1}(\mu_k-\mu_l)
\end{align*}
which is an linear function of $x$. Therefore the decision boundaries are also linear. 
They would be the perpendicular bisectors of the line segments joining the centroids
if $\Sigma=\sigma^2I$. We see the linear discriminant functions
\begin{equation*}
\delta_k=x^T\Sigma^{-1}\mu_k-\frac{1}{2}\mu_k^T\Sigma^{-1}\mu_k+\log \pi_k
\end{equation*}
In practice we estimate the parameters by
\begin{align*}
&\hat{\pi}_k=\frac{N_k}{N}\\
&\hat{\mu}_k=\sum_{g_i=k}x_i/N_k\\
&\hat{\Sigma}=\sum_{k}\sum_{g_i=k}(x_i-\hat{\mu}_k)(x_i-\hat{\mu}_k)^T/(N-K)
\end{align*}
With more than two classes, LDA is not the same as linear regression of
the class indicator matrix, and it avoids the masking problems associated
with that approach. 

When $\Sigma_k$ are not similar, we get \textit{quadratic discriminant functions(QDA)}
\begin{equation*}
\delta_k(x)=-\frac{1}{2}\abs{\Sigma_k}-\frac{1}{2}(x-\mu_k)^T\Sigma_k^{-1}(x-\mu_k)+\log\pi_k
\end{equation*}
Then decision boundary is determines by a quadratic equation. 

Both LDA and QDA perform well on an amazingly large and diverse set
of classification tasks. A reason is
that the data can only support simple decision boundaries such as linear or
quadratic, and the estimates provided via the Gaussian models are stable. 
This is a bias-variance tradeoff. 

\subsection{Regularized Discriminant Analysis}
Regularized covariance matrix: 
\begin{equation*}
\hat{\Sigma}_k(\alpha)=\alpha\hat{\Sigma}_k+(1-\alpha)\hat{\Sigma}
\end{equation*}
where $\hat{\Sigma}$ is the pooled covariance matrix as used in LDA. $\alpha$ 
chosen in validation. 

We can also shrink $\hat{\Sigma}$ toward the scalar matrix
\begin{equation*}
\hat{\Sigma}(\gamma)=\gamma\hat{\Sigma}+(1-\gamma)\hat{\sigma}^2I
\end{equation*}
It leads to a more general family of covariance $\hat{\Sigma}(\alpha,\gamma)$. 

\subsection{Computations for LDA}
Computation of LDA and QDA are simplified by diagonalizing $\hat{\Sigma}_k=U_kD_kU_k^T$. 
Then we can calculate LDA by $X*=D^{-\frac{1}{2}}U^TX$, which makes the covariance 
estimation identity, and classify to the closest class centroid in the transformed space, modulo
the effect of the class prior probabilities $\pi_k$. 

\subsection{Reduced-Rank Linear Discriminant Analysis}
The $K$ centroids in $p$-dimensional space lie in an affine subspace of dimension 
$\le K-1$. When $p$ is much larger then $K$, there would be a large drop. Moreover, 
when locating the closest centroid, we may ignore directions orthogonal to the subspace
since they contribute equally to the subspace. Thus we may as well project the $X^*$ 
to the subspace-spanning subspace $H_{K-1}$, and make distance comparisons there. 
In doing so we would not have relinquished any of the information needed for LDA 
classification.

When $K>3$, we might ask for a $L<K-1$-dimensional subspace $H_L$. Fisher defined optimal to
mean that the projected centroids were spread out as much as possible in terms of 
variance, which means the PC subspaces of the centroids themselves.
In summary, we can find the sequence of optimal subspaces for LDA by
\begin{itemize}
\item Compute $K\times p$ class centroids $M$ and common covariance matrix $W$
\item Compute $M^*=MW^{-\frac{1}{2}}$ using eigen-decomposition of $W$. 
\item Compute $B^*$, covariance matrix of $M^*$ and its eigen-decomposition 
$V^*D_BV^{*T}$. Columns of $V^*$ are the coordinates of the optimal subspace. 
\end{itemize}
The $l$th discriminant variable is given by $Z_l=v_l^TX$ with $v_l=W^{-\frac{1}{2}v_l^*}$. 

\textbf{Fisher's approach}: Find $Z=a^TX$ such that the between class
variance is maximized relative to the within-class variance. The between class 
variance is the variance of the class means of $Z$, and the within class variance is 
the pooled variance about the means. 

The between class variance is $a^TBa$ and within class variance is $a^TWa$. $B$ is
the covariance matrix of the class centroid matrix $M$. Total variance $T=B+W$. 
So Fisher's problem is actually maximizing Rayleigh quotient
\begin{equation*}
\max_a\frac{a^TBa}{a^TWa}
\end{equation*}
We can easily get that $a$ is given by the corresponding eigenvector of the largest 
eigenvalue of $W^{-1}B$. And $a_2,a_3$... $a_l$ are referred to as \textit{discriminant
coordinates/canonical variates}. 

\textbf{A Brief Summary}
\begin{itemize}
\item Gaussian classification with common covariances leads to linear decision
boundaries. Classification can be achieved by sphering the data
with respect to $W$, and classifying to the closest centroid (modulo
$\log\pi_k$) in the sphered space. 
\item Confine the data to the subspace spanned by the centroids in the sphered
space.
\item This subspace can be further decomposed into successively optimal
subspaces in term of centroid separation. This decomposition is identical
to the decomposition due to Fisher.
\end{itemize}
There is a close connection between Fisher’s reduced rank discriminant
analysis and regression of an indicator response matrix. A related fact is that 
if one transforms the original predictors $X$ to $\hat{Y}$, then LDA using $\hat{Y}$ 
is identical to LDA in the original space followed by eigen-decomposition of $\hat{Y}^TY$.

\section{Logistic Regression}
The model arises from the desire to model posterior probabilities of the $K$ classes
via linear functions and ensuring their sum is $1$ and remain in $[0,1]$. 
\begin{equation*}
\log\frac{Pr(G=i|X=x)}{Pr(G=K|X=x)}=\beta_{i0}+\beta_i^Tx
\end{equation*}
It is specified in terms of $K-1$ log-odds or logit transformations. Simple calculation
shows that
\begin{align*}
\theta=&\{\beta_{10}, \beta_1^T, ..., \beta_{(K-1)0}, \beta_{K-1}^T\}\\
p_k(x,\theta):=Pr(G=k|X=x)=&\frac{\exp(\beta_{k0}+\beta_k^Tx)}{1+\sum_{l=1}^{K-1}\exp(\beta_{l0}+
\beta_l^Tx)}\\
Pr(G=K|X=x)=&\frac{1}{1+\sum_{l=1}^{K-1}\exp(\beta_{l0}+\beta_l^Tx)}
\end{align*}

Logistic regression models are used mostly as a data analysis and inference tool, 
where the goal is to understand the role of the input variables in explaining the 
outcome. It is less sensitive to outliers compared with LDA. 

\subsection{Fitting Logistic Regression Models}
Usually by maximum likelihood with conditional maximum likelihood of $G$ given $X$, 
Since $Pr(G|X)$ specifies the conditional distribution, the multinormal distribution
is appropriate. Log-likelihood for $N$ observations is 
\begin{equation*}
l(\theta)=\sum_{i=1}^N logp_{g_i}(x_i;\theta),\quad p_k(x_i;\theta)=
Pr(G=k|X=x_i;\theta)
\end{equation*}
For two-class case, let $y_i=g_i\in\{0,1\}$, $p_1(x;\theta)=p(x;\theta)$, $\beta=\{
\beta_{10}, \beta_1\}$, 
\begin{align*}
l(\beta)=&\sum_{i=1}^N\{y_i\log p(x_i;\beta)+(1-y_i)\log (1-p(x_i;\beta))\}
=\sum_{i=1}^N\{y_i\beta^Tx_i-\log(1+e^{\beta^Tx_i})\}\\
\partial_\beta l(\beta)=&\sum_{i=1}^N x_i(y_i-p(x_i;\beta))=0
\end{align*}
which are $p+1$ nonlinear equation of $\beta$. 
To solve this, we can use Newton-Raphson algorithm, where a single Newton update is
\begin{align*}
\beta^{\text{new}}=&\beta^{\text{old}}-\left(\frac{\partial^2l(\beta)}{\partial\beta
\partial\beta^T}\right)^{-1}\partial_\beta l(\beta)\\
\frac{\partial^2l(\beta)}{\partial\beta\partial\beta^T}=&
-\sum_{i=1}^Nx_ix_i^Tp(x_i;\beta)(1-p(x_i;\beta))
\end{align*}

\textbf{Matrix Form}: Let $p$ be the vector of $p(x_i;\beta^{\text{old}})$, 
$\mathbf{W}$ be a diagonal matrix with $i$th element $p(x_i;\beta^{old})
(1-p(x_i;\beta^{old}))$
\begin{align*} \frac{\partial \ell(\beta)}{\partial \beta} 
    &=\mathbf{X}^{T}(\mathbf{y}-\mathbf{p}) \\ 
\frac{\partial^{2} \ell(\beta)}{\partial \beta \partial \beta^{T}} 
&=-\mathbf{X}^{T} \mathbf{W} \mathbf{X} \\
\beta^{\text {new }} &=\beta^{\text {old }}+\left(\mathbf{X}^{T} 
\mathbf{W} \mathbf{X}\right)^{-1} \mathbf{X}^{T}(\mathbf{y}-\mathbf{p}) \\ 
&=\left(\mathbf{X}^{T} \mathbf{W} \mathbf{X}\right)^{-1} \mathbf{X}^{T} 
\mathbf{W}\left(\mathbf{X} \beta^{\text {old }}+\mathbf{W}^{-1}(\mathbf{y}-
\mathbf{p})\right) \\ &=\left(\mathbf{X}^{T} \mathbf{W} \mathbf{X}\right)^{-1} 
\mathbf{X}^{T} \mathbf{W} \mathbf{z} \\
\mathbf{z}&=\mathbf{X} \beta^{\mathrm{old}}+\mathbf{W}^{-1}(\mathbf{y}-\mathbf{p})
\end{align*}
$\mathbf{z}$ is called \textit{adjusted response}. Each iteration actually solves
\begin{equation*}
\beta^{\text {new }} \leftarrow 
\arg \min _{\beta}(\mathbf{z}-\mathbf{X} \beta)^{T}
 \mathbf{W}(\mathbf{z}-\mathbf{X} \beta)
\end{equation*}
Log-likelihood is concave so it should converge, although overshooting can occur. 

For $K\ge 3$, Newton algorithm can also be expressed as an iteratively reweighted 
least squares algorithm, but with a vector of $K-1$ responses and a nondiagonal 
weight matrix per observation. 

\subsection{Quadratic Approximations and Interface}
\begin{itemize}
\item The weighted residual sum-of-squares is the Pearson chi-square statistic
\begin{equation*}
    \sum_{i=1}^{N} 
    \frac{\left(y_{i}-\hat{p}_{i}\right)^{2}}{\hat{p}_{i}\left(1-\hat{p}_{i}\right)}
\end{equation*}
a quadratic approximation to the deviance.
\item If the model is correct, then maximum-likelihood parameter estimates
$\hat{\beta}$ is consistent(converges to the true $\beta$). 
\item CLT shows that $\hat{\beta}$ converges to $N(\beta,(X^TWX)^{-1})$. 
\end{itemize}

\subsection{$L_1$ Regularized Logistic Regression}
We would maximize
\begin{equation*}
    \max _{\beta_{0}, \beta}\left\{\sum_{i=1}^{N}\left[y_{i}\left(\beta_{0}+\beta^{T}
     x_{i}\right)-\log \left(1+e^{\beta_{0}+\beta^{T} x_{i}}\right)\right]-\lambda 
     \sum_{j=1}^{p}\left|\beta_{j}\right|\right\}
\end{equation*}
We typically do not penalize the intercept term and standardize the predictors for the 
penalty to be meaningful. Lasso criterion is concave and can be solved by nonlinear
programming methods. The score equations for non-zero coefficient variables are actually
\begin{equation*}
    \mathbf{x}_{j}^{T}(\mathbf{y}-\mathbf{p})=\lambda \cdot \operatorname{sign}\left(\beta_{j}\right)
\end{equation*}

\subsection{Logistic Regression or LDA?}
In LDA the linear log-posterior odds between class $k$ and $K$ is a consequence of the 
Gaussian assumption for densities and a common covariance matrix. The linear logistic
by construction also has linear logits. Although they have the exact same form, 
the difference lies in the way the linear coefficients are estimated. The
logistic regression model is more general, in that it makes less assumptions. 
The joint density of $X$ and $G$ is 
\begin{equation*}
    \operatorname{Pr}(X, G=k)=\operatorname{Pr}(X) \operatorname{Pr}(G=k | X)
\end{equation*}
For both LDA and logistic, the second term is
\begin{equation*}
    \operatorname{Pr}(G=k | X=x)=\frac{e^{\beta_{k 0}+\beta_{k}^{T} x}}
    {1+\sum_{\ell=1}^{K-1} e^{\beta_{\ell 0}+\beta_{\ell}^{T} x}}
\end{equation*}
Logistic model leaves $Pr(X)$ as an arbitrary density function. And fits parameters by
maximizing conditional likelihood. Although $Pr(X)$ is ignored, we can think
of this marginal density as being estimated in a fully nonparametric and
unrestricted fashion, using the empirical distribution function which places
mass $1/N$ at each observation.

With LDA we fit the parameters by maximizing full log-likelihood based on the joint 
density
\begin{equation*}
    \operatorname{Pr}(X, G=k)=\phi\left(X ; \mu_{k}, \mathbf{\Sigma}\right) \pi_{k}
\end{equation*}
Here marginal density does play a role: it is the mixture density
\begin{equation*}
    \operatorname{Pr}(X)=\sum_{k=1}^{K} \pi_{k} \phi\left(X ; \mu_{k}, 
    \mathbf{\Sigma}\right)
\end{equation*}
which also involves parameters.

By relying
on the additional model assumptions, we have more information about the
parameters, and hence can estimate them more efficiently (lower variance).

However, observations far from the decision boundary (which are
down-weighted by logistic regression) play a role in estimating the common
covariance matrix, which means that LDA is not robust to gross outliers. 

In practice these assumptions are never correct, and often some of the
components of $X$ are qualitative variables. It is generally felt that logistic
regression is a safer, more robust bet than the LDA model, relying on fewer
assumptions. It is our experience that the models give very similar results,
even when LDA is used inappropriately, such as with qualitative predictors.

\section{Separating Hyperplanes}
\textit{Perceptron}: Compute a linear combination of the input features and return 
the sign of $f(x)=\beta_0+\beta^Tx$. Separating hyperplane is $\beta_0+\beta^Tx=0$. 

For the line $\beta_0+\beta^Tx=0$
\begin{itemize}
\item $\beta/||\beta||$ is the vector normal to the surface of the line. 
\item Signed distance of any point to the line is given by
\begin{equation*}
beta^{* T}\left(x-x_{0}\right)
=\frac{1}{\|\beta\|}\left(\beta^{T} x+\beta_{0}\right)
=\frac{1}{\left\|f^{\prime}(x)\right\|} f(x)
\end{equation*}
\end{itemize}

\subsection{Rosenblatt's Perceptron Learning Algorithm}
Target: minimizing the distance of misclassified points to the decision boundary, 
which is
\begin{align*}
D\left(\beta, \beta_{0}\right)&
=-\sum_{i \in \mathcal{M}} y_{i}\left(x_{i}^{T}\beta+\beta_{0}\right)\\
\partial \frac{D\left(\beta, \beta_{0}\right)}{\partial \beta}&
=-\sum_{i \in \mathcal{M}} y_{i} x_{i} \\ 
\partial \frac{D\left(\beta, \beta_{0}\right)}{\partial \beta_{0}}&
=-\sum_{i \in \mathcal{M}} y_{i}
\end{align*}
Using SGD, a step is taken after each observation is visited, 
\begin{equation*}
\left(\begin{array}{l}{\beta} \\ {\beta_{0}}\end{array}\right) \leftarrow
\left(\begin{array}{l}{\beta} \\ {\beta_{0}}\end{array}\right)+
\rho\left(\begin{array}{c}{y_{i} x_{i}} \\ {y_{i}}\end{array}\right)
\end{equation*}
Here $\rho$ is the learning rate. It will converge to a separating hyperplane in finite
steps when classes are linear separable. 

\noindent\textbf{Problems}: 
\begin{itemize}
\item When the data are separable, there are many solutions, and which
one is found depends on the starting values.
\item The “finite” number of steps can be very large. The smaller the gap,
the longer the time to find it.
\item When the data are not separable, the algorithm will not converge,
and cycles develop. The cycles can be long and therefore hard to detect.
\end{itemize}

\subsection{Optimal Separating Hyperplanes}
Target: Separates the two classes and maximizes the distance to the closest point 
from either class. 

Optimization form: 
\begin{equation*}
\begin{array}{c}{\max _{\beta, \beta_{0}=1} M} \\ 
{\text { subject to } \frac{1}{\|\beta\|}y_{i}\left(x_{i}^{T} \beta+\beta_{0}\right) 
\geq M, i=1, \ldots, N}\end{array}
\end{equation*}
We can just set $\|\beta\|=1/M$, then the problem becomes
\begin{equation*}
\begin{array}{c}{\min _{\beta, \beta_{0}} \frac{1}{2}\|\beta\|^{2}} \\ 
{\text { subject to } y_{i}\left(x_{i}^{T} \beta+\beta_{0}\right) \geq 1, i=1, 
\ldots, N}\end{array}
\end{equation*}
This is a convex problem(quadratic criterion with linear inequality constraints) 
with Lagrange function
\begin{equation*}
    L_{P}=\frac{1}{2}\|\beta\|^{2}-\sum_{i=1}^{N} \alpha_{i}\left[y_{i}
    \left(x_{i}^{T} \beta+\beta_{0}\right)-1\right]
\end{equation*}
to be minimized with respect to $\beta,\beta_0$. Set derivatives to zero, we have
\begin{align*}
\beta &=\sum_{i=1}^{N} \alpha_{i} y_{i} x_{i} ,\qquad
0 =\sum_{i=1}^{N} \alpha_{i} y_{i} \\ 
\text{(Wolfe Dual) }L_{D}&= \sum_{i=1}^{N} \alpha_{i}-\frac{1}{2} 
\sum_{i=1}^{N} \sum_{k=1}^{N} \alpha_{i} \alpha_{k} y_{i} y_{k} x_{i}^{T} x_{k} 
\quad\text { subject to } \alpha_{i} \geq 0 \text { and } \sum_{i=1}^{N} \alpha_{i} 
y_{i}=0
\end{align*}
The solution is obtained by maximizing $L_D$ in the positive orthant, a simpler
convex optimization problem. In addition the solution must satisfy the 
Karush–Kuhn–Tucker conditions, which includes the above equations and
\begin{equation*}
    \alpha_{i}\left[y_{i}\left(x_{i}^{T} \beta+\beta_{0}\right)-1\right]=0\quad \forall i
\end{equation*}
We can also see that $\beta$ is defined as a linear combination of support points 
$x_i$-those points defined to be on the boundary of the slab via $\alpha_i>0$. 

The description of the solution in terms of support points seems to suggest
that the optimal hyperplane focuses more on the points that count,
and is more robust to model misspecification, while LDA
depends on all of the data, even points far away from the decision
boundary. However, the identification of these support
points required the use of all the data. Of course, when the points are real Gaussian, 
LDA will be the optimal, and separating hyperplanes will pay a price
for focusing on the (noisier) data at the boundaries of the classes.

When a separating hyperplane exists, logistic regression fit by maximum likelihood 
will always find it, since the log-likelihood can be driven to 0 in this case. 
Other common points include The coefficient vector is defined by a
weighted least squares fit of a zero-mean linearized response on the input
features, and the weights are larger for points near the decision boundary
than for those further away. 